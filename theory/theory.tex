\documentclass[french]{article}

\usepackage{amsmath, amsfonts, amssymb}
\usepackage[normalem]{ulem}
\usepackage{babel}
\usepackage[utf8]{inputenc}
\usepackage{setspace}
\usepackage[top=2.54cm, bottom=2.54cm, left=1.9cm, right=1.9cm]{geometry}
\usepackage{graphicx}
\usepackage{titlesec}
\usepackage{titling}
\usepackage{multirow}
\usepackage{array}
\usepackage{bbm}
\usepackage[labelfont=bf]{caption}
\usepackage{float}
\usepackage{tabularx}
\usepackage{fancyhdr}
\usepackage{lastpage}
\usepackage{enumitem}
\usepackage[toc]{appendix}
\usepackage{tikz}
\usepackage{standalone}
\usepackage{lipsum}

\onehalfspacing

\setlength{\parskip}{0.5cm}

\renewcommand{\vec}[1]{\boldsymbol{#1}}
\newcommand{\grad}[2]{\frac{\partial #1}{\partial #2}}
\newcommand{\ind}[1]{\mathbbm{1}_{\{#1\}}}
\newcommand{\gru}{\text{GRU}}

\title{Dynamic Memory Networks}
\author{}
\date{}

\begin{document}

\maketitle

\section{GRU}

\begin{equation}
  \vec{Z}_t = \sigma\left(\vec{X}_t\vec{W}^{(z)} + \vec{H}_{t-1}\vec{U}^{(z)} + (\vec{b}^{(z)})^T\right)
\end{equation}

\begin{equation}
  \vec{R}_t = \sigma\left(\vec{X}_t\vec{W}^{(r)} + \vec{H}_{t-1}\vec{U}^{(r)} + (\vec{b}^{(r)})^T\right)
\end{equation}

\begin{equation}
  \tilde{\vec{H}}_t = \tanh\left(\vec{X}_t\vec{W} + \vec{R} \odot \vec{H}_{t-1}\vec{U} + \vec{b}^T\right)
\end{equation}

\begin{equation}
  \vec{H}_t = \vec{Z}_t \odot \vec{H}_{t-1} + \left(\vec{1} - \vec{Z}_t\right) \odot \tilde{\vec{H}}_t
\end{equation}

\begin{equation}
  \vec{H}_0 = \vec{0}
\end{equation}

Let $\vec{H}_t = \gru(\vec{X}_t, \vec{H}_{t-1})$ resume the previous equations.


\end{document}